\documentclass{beamer}
\usepackage[utf8]{inputenc}
\usepackage[T1]{fontenc}
\usepackage{lipsum, lmodern}
\usepackage[scaled=.95]{helvet}% helvetica as the origin of arial
\usepackage[helvet]{sfmath}    % for the mathematical enviroments
\usepackage{graphicx}
\usepackage{multimedia}% for add a movie
\usepackage{hyperref}
\usepackage{media9} %for movie : new try
\renewcommand{\familydefault}{\sfdefault}

%% ETH beamer theme
% Options: [default]
%   itemsblack/[itemsblue]: change color of bullets etc. to black/blue in itemize style environments
%   [titlesblack]/titlesblue: change color of frame titles/subtitles to black/blue
% \usetheme[itemsblack,titlesblack]{eth}
\usetheme{eth}

%% Theme uses ETH blue color by default. Can be changed to any color using this command: 
% \setbeamercolor{structure}{fg=blue}

%% Mandatory variables
\author{Jasmin Fischli, Philipp Göldlin, Julie Veya, Jonathan Burkhard}
\title{Paper presentation : UltraStereo: Efficient Learning-based Matching for Active Stereo Systems}
\date{\today}

%% Optional variables
 \supervisor{Sattler Torsten} % for one supervisor
%\supervisors{Carol Foote, Jane Smith} % for multiple supervisors
% \projecttype{Master's Thesis}

\begin{document}
\frame{\maketitle}
\begin{frame}{Table of contents}{}
	\tableofcontents
\end{frame}

%Camera support
\section{Slide Example}
\begin{frame}{Example}

\begin{figure}
\includegraphics[scale=0.25]{pictures/polop}

\caption{Example}
\end{figure}
\end{frame}
\section{Evaluation}
\subsection{???}
\begin{frame}{???}
\end{frame}

\subsection{Invalidation}
\begin{frame}{Invalidation}
\begin{figure}
\includegraphics[scale=0.08]{pictures/fig3}
\caption{Quantitative results on syntatic data}
\end{figure}
\end{frame}

\begin{frame}{Example of depth-map produced with UltraStereo}
\begin{figure}
\includegraphics[scale=0.08]{pictures/fig5}
\caption{Qualitative Evaluation}
\end{figure}
\end{frame}

\begin{frame}{Edge fattening}
\begin{figure}
\includegraphics[scale=0.15]{pictures/fig6}
\caption{???}
\end{figure}
\end{frame}

\subsection{Binary representation}
\begin{frame}{Binary representation}
\begin{figure}
\includegraphics[scale=0.1]{pictures/fig7}
\caption{Quantitative results on syntatic data}
\end{figure}
\end{frame}

\subsection{Interferenc and Generalization}
\begin{frame}{Interferenc and Generalization}
\begin{figure}
\includegraphics[scale=0.08]{pictures/fig8}
\caption{Quantitative results on syntatic data}
\end{figure}
\end{frame}

\section{Conclusion}
\begin{frame}{Conclusion}
Best algorithms ever made !
\end{frame}

\begin{frame}{}
\centering
\Huge{Thank you for your attention !}
\end{frame}



\end{document}
%Document end *************************************************

\section{Introduction}
\subsection{One subsection}
\begin{frame}{Itemize}
\begin{itemize}
\item Here you can see an itemization
\begin{itemize}
\item It has items
\begin{itemize}
\item The items are below each other
\end{itemize}
\end{itemize}
\end{itemize}
\end{frame}

\begin{frame}{Enumerate}
\begin{enumerate}
\item Here you can see an enumeration
\item It has items
\item The items are numbered
\end{enumerate}
\[
	f(x)=\sum_{i=0}^\infty \frac{f^{(i)}(x_0)}{i!}(x-x_0)^i
\]
\end{frame}

\subsection{New Subsection}
\begin{frame}{Theorems and environments}
\begin{theorem}[Sample theorem]
This presentation is essentially useless.
\end{theorem}
\begin{proof}
This proof is essentially incorrect.
\end{proof}
\end{frame}

\begin{frame}{Example slide with Title}
\begin{example}
Major problem.
\end{example}
\begin{solution}
Minor nuisance.
\end{solution}
\end{frame}

\begin{frame}[plain]{Plain frame with title}
\lipsum[1]
\end{frame}


